\frame{
\frametitle{Collaborated work}
\vspace{-1ex}
\bluebox{I worked on other projects}
{
\begin{itemize}
    \item Developed a {\bf data-informed active subspace regularization framework} for solving inverse problems [\myblue{6}]. In DIAS, the observation data informs the active subspace of PoI. We keep this active subspace intact from regularization, thus achieving better inverse solutions compared to Tikhonov regularization framework.
    \item Developed {\bf model-Constrained machine learning approaches for solving Bayesian inverse problems} [\myblue{5}]. We aim to develop a real-time uncertainty qualification approach for inverse problems. This work was built upon the \texttt{TNet} [\myblue{4}].
    \item {\bf Unifying randomized methods for inverse problems} [\myblue{7}]. We unified different randomized methods for solving inverse problems under the same umbrella and thus providing a guideline for discovering new randomization methods.
\end{itemize}
}

% \vspace{1ex}

{ \tiny
[4] {\bf H.V. Nguyen}, et. al. {\it ``TNet: A Model-Constrained Deep Learning Approaches for Inverse Problems."} SIAM Journal of Scientific
Computing (2024) \newline
[5] R.S. Philley, {\bf H.V. Nguyen}, et. al. {\it ``Model-Constrained Empirical Bayesian Neural Networks for Inverse Problems."} XLIV
Ibero-Latin ACCME (2023) \newline
[6] {\bf H.V. Nguyen}, et. al. {\it ``A Data-Informed Active Subspace Regularization Framework for Inverse Problems."} Computation (2022) \newline
[7] J. Wittmer, {\bf H.V. Nguyen}, et. al. {\it ``On Unifying Randomized Methods for Inverse Problems."} Inverse Problems (2023)
}

}