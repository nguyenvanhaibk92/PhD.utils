\frame{
\frametitle{Numerical results - X-ray tomography inverse problem}
\vspace{-4ex}
\bluebox{Why don't we use SVD of $G$ for unimportant modes of parameter $\ub$?}
{

\begin{figure}[htb!]
    \centering
    % \caption{{\bf 2D heat equation.} A (random) representative case of inverse and full forward solution obtained by $\TNetAE$ trained with 1 training sample coupled with data randomization of noise level $\sigma = 0.1$. $\TNetAE$ inverse solution is comparable to the Tikhonov (Tik) inverse counterpart, and both are consistent with the ground truth (True). $\TNetAE$ full forward solution is almost identical (in fact within 3 digits of accuracy) to the underlying true solution.}
    \resizebox{\textwidth}{!}{
    \begin{tabular*}{\textwidth}{c@{\hskip 0.4cm} c@{\hskip 0.4cm} c@{\hskip 0.4cm}}
        \centering
        \raisebox{-0.5\height}{1st eigenvector of $G$} &
        \raisebox{-0.5\height}{1st eigenvector of AS} &
        \raisebox{-0.5\height}{true parameter $\ub$}
        \\ ~\\
        \raisebox{-0.5\height}{\includegraphics[width = 0.3\textwidth]{Figs/Figs/DIAS/EigenV_1.pdf}} &
        \raisebox{-0.5\height}{\includegraphics[width = 0.3\textwidth]{Figs/Figs/DIAS/EigenW_1.pdf}} &
        \raisebox{-0.5\height}{\includegraphics[width = 0.3\textwidth]{Figs/Figs/DIAS/X_ray_True.pdf}}
    \end{tabular*}
    }
\end{figure}

}

}